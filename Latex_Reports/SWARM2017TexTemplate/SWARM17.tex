%%%%%%%%%%%%%%%%%%%%%%%%%%%%%%%%%%%%%%%%%%%%%%%%%%%%
\documentclass[fleqn,10pt,twocolumn]{SWARM17}

\title{Preparation of Papers in a Two-Column Format for\\
       SWARM 2017 with \LaTeX }

\author{Taro Mure${}^{1\dagger}$ and Hanako Gun${}^{2}$}
% The dagger symbol indicates the presenter.
\speaker{Taro Mure}

\affils{${}^{1}$Department of Xxx, University of Xxxx, Kyoto, Japan\\
(Tel: +81-2-000-0000; E-mail: taro@xxx.ac.jp)\\
${}^{2}$Department of Yyy, Yyyy University, Kyoto, Japan\\
(Tel: +81-3-000-0000; E-mail: hanako@yyyy.ac.jp)\\
}
\abstract{%
This template provides a sample format of final manuscripts
for SWARM 2017.  It also provides instructions to
authors for completing the final manuscripts which will be
included in the conference Proceedings.
This template is available for abstracts, short and full papers. }

\keywords{%
Selected keywords relevant to the subject.
}

\begin{document}

\maketitle

%-----------------------------------------------------------------------

\section{Introduction}

Each paper must be divided into two parts. The first part includes the
title, authors' name, abstract and keywords. The second part is the
main body of the paper.

\section{Page size and format}

The size of the manuscript must be A4 and should be typed in two columns with single spacing.
%
There are three options for submission: full paper, short paper or abstract.
%
The only difference between the formats of these options is the number of maximum pages.
%
Full papers and short papers have an 8-page or a 4-page maximum length and should report on new unpublished work. 
%
Surveys are also welcomed as full papers. 
%
Abstracts are limited to 2 pages and can report on previously published work with the expectation of offering new viewpoints on that work.
%
%Up to two extra pages are permitted.
%
%For each page exceeding the standard length, over-length page fee will be charged.
%
See conference website for detail.

%\begin{table}
\begin{center}
\begin{tabular}{lc}
    Left margin  &  20mm \\
    Right margin &  20mm \\
    Top margin   &  25mm \\
    Bottom margin&  25mm \\
    Column width &  80mm \\
\end{tabular}
\end{center}
%\end{table}

Submitted papers MUST BE in Portable Document Format (PDF).  NO OTHER
FORMATS WILL BE ACCEPTED.  The size of the PDF file to be sent
electronically should not exceed four megabytes (4 MB), regardless of
the number of pages.  

\section{Fonts and style}

\subsection{First part}

The first part includes the paper title, authors' name, abstract, and
keywords, all of them in Times Roman or similar font.  The font
size of the title, authors' name, affiliation, abstract, and keywords
are bold 12pt, 11pt, 10pt, 10pt, and 10pt, respectively.

\subsection{Paper body}

The second part consisting of the body of the paper must be edited in
double column format, with each column 80mm width and separated by
10mm.  The top-level heading, usually called section, numbered in
Arabic numerals, shall appear centered on the column with Times Roman
capital bold 11pt.  The numbered level-two heading starts from the
left in Times Roman bold 10pt font.  The main text uses Times Roman
10pt font with single spacing.  The first line of a new paragraph is
indented by 4mm.

\section{Figures, tables, and equations}

\subsection{Figures and tables}

Place figures and tables at the top or bottom of columns.
Avoid placing them before their first mention in the text.
Large figures and tables may span across both columns.
Scanned images (e.g., line art, photos) can be used if the output
resolution is at least 600dpi.

\begin{table}[b]
\caption{Caption should be placed above the table.}
\begin{center}
\begin{tabular}{|c|c|c|c|}\hline
    &   A   &   B   &   C \\\hline
(1) & 150\% & 16.3\% & 18.2\% \\\hline
\end{tabular}
\end{center}
\end{table}


\begin{figure}[b]
\begin{center}
\includegraphics[width=8cm]{SWARM17_fig.eps}
\caption{\label{test} Caption should be placed below the figure.}
\end{center}
\end{figure}

Figure captions should be below the figures; table captions should
be above the tables.  They should be referred to in the text, for
example, Fig. \ref{test}, or Figs. 1 to 3.

\subsection{Equations}

Equation numbers should be Arabic numerals enclosed in parentheses on
the right-hand margin.  They should be cited in the text as, for
example, Eq. (\ref{eq:ss}), or Eqs. (1) to (3).  Equations start
from the left of the column.  Punctuate equations with commas or
periods when they are part of a sentence.  For example,

\begin{eqnarray}
\begin{array}{rcl}
\dot{x}&=&Ax+Bu,\\
y&=&Cx+Du,
\end{array}
\label{eq:ss}
\end{eqnarray}
where $x$ is the state vector.

\subsection{References}

References should appear in a separate bibliography at the end of the
paper, with items referred to with numerals in square brackets
\cite{ref1,ref3,ref4,ref5}.  Times Roman 10pt is used for references.

\section{PAGE NUMBERS}

Do not put page numbers in the manuscript PDF.

%%%%%%%%%%%%%%%%% BIBLIOGRAPHY IN THE LaTeX file !!!!! %%%%%%%%%%%%%%%%%%%%%%
\begin{thebibliography}{9}
\bibitem{ref1}
SWARM 2017 Website,\\ ``http://www.ohk.hiroshima-u.ac.jp/SWARM2017/''

\bibitem{ref2}
T. Mure, ``XXX Control of Nonlinear Systems'',
{\it International Journal of YYY}, Vol. 99, No. 99, pp. 123--456, 2012.

\bibitem{ref3}
H. Gun, {\it ZZZ Handbook}, SICE, Tokyo, 2012.

\bibitem{ref4}
T. Mure, ``Measurement Method using ABC Sensor'', {\it Transaction of EFG}, 
Vol. 00, No. 1, pp. 1--9, 2011.

\bibitem{ref5}
H. Gun, ``SSS Intelligent Systems'', {\it International Journal of VVV}, 
Vol. 12, No. 5, pp. 500--600, 2011.

\end{thebibliography}

\end{document}

